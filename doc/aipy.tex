\documentclass[10pt]{article}
\usepackage{fullpage}
\usepackage{amsmath}
\usepackage{graphicx}

\title{Tutorial on Astronomical Interferometry in PYthon (AIPY)}

\author{Aaron Parsons}
\date{13 December 2007}

\begin{document}
\maketitle
\tableofcontents 

\section{Introduction}

This package collects together tools for radio astronomical interferometry.  In
addition to pure-python phasing, calibration, imaging, and
deconvolution code, this package includes interfaces to MIRIAD (a Fortran
interferometry package), HEALPix (a package for representing spherical data
sets), routines from SciPy for fitting, and the PyFITS and PyEphem packages
verbatim.  It can be found on the web at
{\it http://setiathome.berkeley.edu/\~{}aparsons/aipy}

AIPY is under active development as a part of the NSF-funded project
PAPER: the Precision Array for Probing the Epoch of Reionization, and 
low-frequency interferometry experiment for detecting ionization resulting
from the formation of the first starts and galaxies.

AIPY is free software; you can redistribute it and/or modify it under
the terms of the GNU General Public License as published by the Free Software
Foundation; either version 2 of the License, or (at your option) any later
version.

\subsection{Acknowledgement of Intellectual Debt}

The subpackage "optimize" is released under GPL by SCIPY developers, and
has been modified to include only pure-python implementations.

The package "pyephem" is released under GPL by Brandon Craig Rhodes, and
is included in AIPY with minor modification to setup.py.

The package "pyfits" is released for redistribution and modification by
the Association of Universities for Research in Astronomy (AURA), and
is included in AIPY with minor modification to setup.py and \_\_init\_\_.py.

The C source code for "healpix" (in healpix/cxx) is released under GPL
by the HEALPix collaboration, and is included in AIPY verbatim.

The C/FORTRAN source code for "miriad" (in miriad/mirsrc) is released under
GPL. It is the work of many authors, and is included in AIPY verbatim.

\section{Installation}

\subsection{Requirements}

\begin{itemize}
\item[] *nix or MacOs.  I'm pretty certain this doesn't install on Windows.
\item[] Python 2.4 or better.  {\it http://www.python.org}
\item[] Numpy 1.0.4 or better. {\it http://numpy.scipy.org}
\item[] Matplotlib 90.0 or better. {\it http:://matplotlib.sourceforge.net}
\item[] Matplotlib-Basemap 0.9.0 or better
        {\it http:://matplotlib.sourceforge.net}
\end{itemize}

\subsection{Install As Root}

\begin{verbatim}
$ sudo python setup.py install
\end{verbatim}

\subsection{Install As User}

\begin{verbatim}
$ python setup.py install --install-lib <module_dir> --install-scripts <scripts_dir>
\end{verbatim}

This puts the python module in <module\_dir> and the command-line scripts
in <scripts\_dir>.  The next thing is to tell python where to look
for the AIPY module.  This is done by setting the PYTHONPATH shell variable
to point to <module\_dir>.  If you are using bash, add the following line
to you .bashrc file:
\begin{verbatim}
export PYTHONPATH=PYTHONPATH:<module_dir>
\end{verbatim}


\input miriad.tex
\input ant.tex
% \input sect4.tex
% \input sect5.tex

%\bibliographystyle{plain}   % or "unsrt", "alpha", "abbrv", etc.
%\bibliography{biblio}       % use data in file "biblio.bib"
\end{document}


