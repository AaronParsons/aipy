\section{Module: aipy.deconv}

A module implementing various techniques for deconvolving an image by a
kernel.  Currently implemented are Clean, Least-Squares, and Maximum Entropy.

\subsection{Function: clean}

This is an implementation of the standard H{\"o}gbom clean deconvolution
algorithm \cite{hogbom1974}, which operates on the assumption that the image
is composed of point sources.  This makes it a poor choice for images with
distributed flux.  The algorithm works by iteratively constructing a model.
Each iteration, a point is added to this model at the location of the maximum
residual, with a fraction (specified by 'gain') of the magnitude.  The
convolution of that point is removed from the residual, and the process
repeats.  Termination happens after 'maxiter' iterations, or when the clean
loops starts increasing the magnitude of the residual.

\subsection{Function: lsq}

Implements a simple least-square fitting procedure for deconvolving an image.
However, to save computing, the gradient of the fit at each pixel with respect
pixels in the image is approximated as being diagonal.  In essence, this
assumes that the convolution kernel is a delta-function.  This assumption
works for small kernels, but not so well for large ones
\cite{cornwell_evans1985}.  This deconvolution algorithm, unlike maximum
entropy, makes no promises about maximizing smoothness while fitting to the
expected noise and flux levels.  That is, it can introduce structure for which
there is no evidence in the original image.  Termination happens after
'maxiter' iterations, or when the score is changing by a fraction less than
'tol' between iterations.  This makes the assumption that the true optimum has
a smooth approach.

\subsection{Function: maxent}

The maximum entropy deconvolution (MEM) \cite{cornwell_evans1985,sault1990} is
similar to least-squares deconvolution, but instead of simply minimizing the
fit, maximum entropy seeks to do so only to within the specified variance
var0, and then attempts to maximize the "smoothness" of the model.  This has
several desirable effects, including uniqueness of solution, independence from
flux distribution (all image scales are equally weighted in the model), and
absence of spurious structure in the model.  Similar approximations are made
in this implementation as in the least-squares implementation.

